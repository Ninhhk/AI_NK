\documentclass[../DoAn.tex]{subfiles}
\begin{document}

\section{Kết luận}

\subsection{So sánh với các sản phẩm tương tự}

Để đánh giá khách quan kết quả đạt được, đồ án thực hiện so sánh hệ thống AI NVCB với các sản phẩm AI phổ biến trên thị trường hiện nay:

\begin{table}[H]
\centering
\caption{So sánh AI NVCB với các sản phẩm tương tự}
\label{tab:comparison}
\begin{tabular}{|l|c|c|c|c|c|}
\hline
\textbf{Tiêu chí} & \textbf{ChatGPT} & \textbf{Gemini} & \textbf{Gamma.app} & \textbf{Beautiful.ai} & \textbf{AI NVCB} \\ \hline
\textbf{Phân tích tài liệu} & Giới hạn & ✓ & ✗ & ✗ & ✓ Đa định dạng \\ \hline
\textbf{Tạo slide tự động} & ✗ & ✗ & ✓ & ✓ & ✓ \\ \hline
\textbf{Tạo câu hỏi trắc nghiệm} & Thủ công & Thủ công & ✗ & ✗ & ✓ Tự động \\ \hline
\textbf{Hỏi đáp RAG} & Trả phí & ✓ & ✗ & ✗ & ✓ \\ \hline
\textbf{Triển khai On-Premise} & ✗ & ✗ & ✗ & ✗ & \textbf{✓} \\ \hline
\textbf{Tùy chỉnh Model} & ✗ & ✗ & ✗ & ✗ & \textbf{✓} \\ \hline
\textbf{Chi phí} & Cao & Trung bình & Cao & Cao & \textbf{Thấp/Miễn phí} \\ \hline
\textbf{Bảo mật dữ liệu} & Cloud & Cloud & Cloud & Cloud & \textbf{On-premise} \\ \hline
\textbf{Hỗ trợ tiếng Việt} & Tốt & Tốt & Trung bình & Kém & \textbf{Tối ưu hóa} \\ \hline
\end{tabular}
\end{table}

\textbf{Phân tích so sánh:}

\begin{enumerate}
\item \textbf{Ưu điểm vượt trội của AI NVCB:}
\begin{itemize}
\item \textbf{Triển khai On-Premise hoàn toàn}: Là sản phẩm duy nhất cho phép chạy 100\% offline, đảm bảo an toàn dữ liệu tuyệt đối
\item \textbf{Tích hợp đa chức năng}: Kết hợp phân tích tài liệu, tạo slide và tạo quiz trong một hệ thống duy nhất
\item \textbf{Chi phí thấp}: Sử dụng các LLM mã nguồn mở miễn phí thay vì API trả phí
\item \textbf{Tùy chỉnh linh hoạt}: Cho phép hot-swap giữa các model và tùy chỉnh system prompt
\end{itemize}

\item \textbf{Hạn chế so với các sản phẩm thương mại:}
\begin{itemize}
\item Chất lượng output phụ thuộc vào model được chọn (7B-14B parameters)
\item Yêu cầu phần cứng tương đối cao để chạy local
\item Giao diện đơn giản hơn so với các sản phẩm chuyên biệt như Gamma.app
\end{itemize}
\end{enumerate}

\subsection{Tổng kết kết quả đạt được}

Trong quá trình thực hiện đồ án tốt nghiệp, sinh viên đã hoàn thành được các công việc sau:

\subsubsection{Các chức năng đã hoàn thiện}

\begin{table}[H]
\centering
\caption{Trạng thái hoàn thiện các Use Case}
\label{tab:usecase_completion}
\small
\begin{tabular}{|l|p{5cm}|c|p{3cm}|}
\hline
\textbf{Use Case} & \textbf{Mô tả} & \textbf{Trạng thái} & \textbf{Ghi chú} \\ \hline
UC01.1 & Upload tài liệu đa định dạng & ✅ Hoàn thành & PDF, DOCX, TXT, MD \\ \hline
UC01.2 & Tóm tắt nội dung tài liệu & ✅ Hoàn thành & Đơn lẻ \& đa tài liệu \\ \hline
UC01.3 & Hỏi đáp với RAG & ✅ Hoàn thành & FAISS vector search \\ \hline
UC01.4 & Lưu trữ lịch sử hội thoại & ✅ Hoàn thành & SQLite persistence \\ \hline
UC02 & Tạo slide PowerPoint & ✅ Hoàn thành & Preview \& export PPTX \\ \hline
UC03 & Tạo câu hỏi trắc nghiệm & ✅ Hoàn thành & Multiple choice + giải thích \\ \hline
UC04.1 & Liệt kê model có sẵn & ✅ Hoàn thành & Ollama API integration \\ \hline
UC04.2 & Tải model mới & ✅ Hoàn thành & Pull from Ollama registry \\ \hline
UC04.3 & Xóa model & ✅ Hoàn thành & Delete với xác nhận \\ \hline
UC04.4 & Thiết lập model mặc định & ✅ Hoàn thành & Singleton pattern \\ \hline
UC04.5 & Tùy chỉnh system prompt & ✅ Hoàn thành & JSON config persistence \\ \hline
\end{tabular}
\end{table}

\subsubsection{Các chỉ tiêu phi chức năng đạt được}

\begin{table}[H]
\centering
\caption{Đánh giá các chỉ tiêu phi chức năng}
\label{tab:nfr_results}
\begin{tabular}{|l|c|c|c|}
\hline
\textbf{Chỉ tiêu} & \textbf{Yêu cầu (NFR)} & \textbf{Thực tế} & \textbf{Đánh giá} \\ \hline
Thời gian tóm tắt (<10MB) & <30 giây & 18 giây & ✅ Đạt \\ \hline
Thời gian tạo slide (10 slide) & <60 giây & 35 giây & ✅ Đạt \\ \hline
Thời gian tạo quiz (10 câu) & <45 giây & 28 giây & ✅ Đạt \\ \hline
Số người dùng đồng thời & ≥10 & 10+ & ✅ Đạt \\ \hline
Độ sẵn sàng hệ thống & 99\% & 99.5\% & ✅ Đạt \\ \hline
Độ chính xác output tiếng Việt & - & 99.5\% & ✅ Tốt \\ \hline
Tỷ lệ parse JSON thành công & - & 99\% & ✅ Tốt \\ \hline
\end{tabular}
\end{table}

\subsubsection{Thống kê mã nguồn}

\begin{table}[H]
\centering
\caption{Thống kê mã nguồn hệ thống}
\label{tab:code_stats}
\begin{tabular}{|l|r|}
\hline
\textbf{Thành phần} & \textbf{Số liệu} \\ \hline
Tổng dòng code Python & 12,398 \\ \hline
Số file Python & 49 \\ \hline
Số package/module & 10 \\ \hline
Dung lượng project & \textasciitilde180 MB \\ \hline
Số test case & 17 \\ \hline
Tỷ lệ pass test & 100\% \\ \hline
\end{tabular}
\end{table}

\subsection{Những hạn chế còn tồn tại}

Mặc dù đã hoàn thành các chức năng chính, hệ thống vẫn còn một số hạn chế cần được cải thiện:

\begin{enumerate}
\item \textbf{Yêu cầu phần cứng cao:}
\begin{itemize}
\item Cần tối thiểu 8GB RAM để chạy model 7B parameters
\item Khuyến nghị có GPU để đạt hiệu suất tốt hơn
\item Không phù hợp với các máy tính cấu hình thấp
\end{itemize}

\item \textbf{Chưa có hệ thống xác thực người dùng:}
\begin{itemize}
\item Hiện tại thiết kế cho single-user hoặc mạng nội bộ tin cậy
\item Chưa có phân quyền và quản lý tài khoản
\item Không phù hợp cho triển khai công khai trên Internet
\end{itemize}

\item \textbf{Giao diện slide hạn chế:}
\begin{itemize}
\item Chỉ có các template PowerPoint cơ bản
\item Chưa hỗ trợ tùy chỉnh màu sắc, font chữ nâng cao
\item Không có tính năng kéo thả (drag-and-drop)
\end{itemize}

\item \textbf{Chưa tối ưu cho thiết bị di động:}
\begin{itemize}
\item Giao diện Streamlit thiết kế cho desktop
\item Trải nghiệm trên mobile và tablet chưa tốt
\end{itemize}

\item \textbf{Chưa hỗ trợ cộng tác:}
\begin{itemize}
\item Chỉ cho phép một người dùng chỉnh sửa
\item Không có tính năng chia sẻ và cộng tác real-time
\end{itemize}

\item \textbf{Xử lý bất đồng bộ chưa hoàn thiện:}
\begin{itemize}
\item Một số tác vụ LLM dài có thể block UI
\item Cần cải thiện progress tracking và background processing
\end{itemize}
\end{enumerate}

\subsection{Các đóng góp nổi bật}

Đồ án đã đạt được những đóng góp có giá trị thực tiễn cao:

\begin{table}[H]
\centering
\caption{Các đóng góp nổi bật của đồ án}
\label{tab:contributions}
\small
\begin{tabular}{|c|p{3.5cm}|p{4.5cm}|p{3cm}|}
\hline
\textbf{STT} & \textbf{Đóng góp} & \textbf{Mô tả} & \textbf{Tác động} \\ \hline
1 & \textbf{Nền tảng AI On-Premise} & Hệ thống hoạt động hoàn toàn offline với Ollama, không phụ thuộc cloud & 100\% bảo mật dữ liệu \\ \hline
2 & \textbf{RAG Pipeline tiếng Việt} & Truy xuất theo chủ đề với chunking tối ưu cho tiếng Việt & 84\% F1-Score \\ \hline
3 & \textbf{Hot-Swap LLM Management} & Chuyển đổi model runtime không cần khởi động lại & <2 giây chuyển đổi \\ \hline
4 & \textbf{Xử lý ngôn ngữ Việt} & Bộ lọc ký tự Trung Quốc + enforcement qua system prompt & 99.5\% output thuần Việt \\ \hline
5 & \textbf{JSON Structured Generation} & Logic retry với error context cho nội dung có cấu trúc & 99\% tỷ lệ thành công \\ \hline
6 & \textbf{Production Deployment} & Docker multi-stage build, health monitoring & Giảm 68\% kích thước image \\ \hline
\end{tabular}
\end{table}

\subsection{Bài học kinh nghiệm}

Trong suốt quá trình thực hiện đồ án, sinh viên đã rút ra được nhiều bài học quý giá:

\subsubsection{Về kỹ thuật}

\begin{enumerate}
\item \textbf{AI On-Premise khả thi cho ứng dụng giáo dục:}
\begin{itemize}
\item Ollama kết hợp các LLM mã nguồn mở (Qwen2.5, Llama3, Gemma2) cung cấp chất lượng đủ tốt cho các tác vụ giáo dục
\item Chi phí vận hành thấp hơn đáng kể so với sử dụng API cloud
\item Phù hợp với các tổ chức có yêu cầu cao về bảo mật dữ liệu
\end{itemize}

\item \textbf{Thách thức xử lý tiếng Việt với LLM:}
\begin{itemize}
\item Các model đa ngôn ngữ thường trộn lẫn tiếng Việt với tiếng Trung
\item Cần áp dụng giải pháp nhiều lớp: prompt engineering + character filtering + validation
\item System prompt rõ ràng và nhất quán là chìa khóa để đảm bảo chất lượng output
\end{itemize}

\item \textbf{Structured Output từ LLM cần robust handling:}
\begin{itemize}
\item LLM không luôn trả về JSON hợp lệ trong lần đầu
\item Cần logic retry với error context để AI tự sửa lỗi
\item Parsing linh hoạt với multiple fallback strategies
\end{itemize}

\item \textbf{Tối ưu RAG cho tiếng Việt:}
\begin{itemize}
\item Chunk size (1000 tokens) và overlap (200 tokens) là tham số quan trọng
\item Embedding model \texttt{all-MiniLM-L6-v2} cho kết quả tốt với tiếng Việt
\item Topic-based retrieval hiệu quả hơn pure semantic search
\end{itemize}
\end{enumerate}

\subsubsection{Về kiến trúc phần mềm}

\begin{enumerate}
\item \textbf{Clean Architecture tạo nền tảng bảo trì tốt:}
\begin{itemize}
\item Phân tách rõ ràng các layer (Presentation, API, Business Logic, Data Access)
\item Repository Pattern giúp dễ dàng thay đổi database engine
\item Singleton Pattern đảm bảo consistency cho global state
\end{itemize}

\item \textbf{API-first design:}
\begin{itemize}
\item FastAPI với auto-documentation giúp frontend và backend phát triển song song
\item Swagger UI hỗ trợ testing và debug hiệu quả
\item RESTful design dễ mở rộng và tích hợp
\end{itemize}

\item \textbf{Docker hóa từ đầu:}
\begin{itemize}
\item Multi-stage build giảm đáng kể kích thước production image
\item Docker Compose đơn giản hóa deployment
\item Health check endpoints quan trọng cho monitoring
\end{itemize}
\end{enumerate}

\subsubsection{Về quản lý dự án}

\begin{enumerate}
\item \textbf{Iterative development hiệu quả:}
\begin{itemize}
\item Bắt đầu với MVP (Minimum Viable Product) rồi mở rộng dần
\item Feedback sớm giúp điều chỉnh hướng đi kịp thời
\item Documentation song song với development
\end{itemize}

\item \textbf{Testing là đầu tư xứng đáng:}
\begin{itemize}
\item Unit tests giúp phát hiện regression sớm
\item Smoke tests đảm bảo deployment thành công
\item Test coverage nên được duy trì từ đầu dự án
\end{itemize}
\end{enumerate}

\section{Hướng phát triển}

\subsection{Công việc hoàn thiện các chức năng hiện có}

Để nâng cao chất lượng và trải nghiệm người dùng, các công việc cần thiết trong ngắn hạn bao gồm:

\subsubsection{Hoàn thiện chức năng phân tích tài liệu}

\begin{table}[H]
\centering
\caption{Kế hoạch hoàn thiện phân tích tài liệu}
\label{tab:document_enhancement}
\begin{tabular}{|p{4cm}|p{6cm}|c|}
\hline
\textbf{Công việc} & \textbf{Mô tả} & \textbf{Độ ưu tiên} \\ \hline
Hỗ trợ thêm định dạng & Excel, CSV, HTML, LaTeX & Cao \\ \hline
Cải thiện chunking & Adaptive chunk size theo ngữ cảnh & Trung bình \\ \hline
OCR tích hợp & Trích xuất text từ hình ảnh trong PDF & Cao \\ \hline
Citation tracking & Truy xuất nguồn gốc thông tin trong output & Trung bình \\ \hline
\end{tabular}
\end{table}

\subsubsection{Nâng cấp tính năng tạo slide}

\begin{table}[H]
\centering
\caption{Kế hoạch nâng cấp tạo slide}
\label{tab:slide_enhancement}
\begin{tabular}{|p{4cm}|p{6cm}|c|}
\hline
\textbf{Công việc} & \textbf{Mô tả} & \textbf{Độ ưu tiên} \\ \hline
Thêm template mới & 10+ template chuyên nghiệp & Cao \\ \hline
Tùy chỉnh theme & Color picker, font selection & Trung bình \\ \hline
Hỗ trợ hình ảnh & Tự động chèn hình minh họa & Cao \\ \hline
Export đa định dạng & PDF, Google Slides, Keynote & Trung bình \\ \hline
\end{tabular}
\end{table}

\subsubsection{Cải thiện tạo câu hỏi trắc nghiệm}

\begin{table}[H]
\centering
\caption{Kế hoạch cải thiện tạo câu hỏi}
\label{tab:quiz_enhancement}
\begin{tabular}{|p{4cm}|p{6cm}|c|}
\hline
\textbf{Công việc} & \textbf{Mô tả} & \textbf{Độ ưu tiên} \\ \hline
Đa dạng loại câu hỏi & True/False, Fill-in-blank, Matching & Cao \\ \hline
Điều chỉnh độ khó & Easy, Medium, Hard levels & Cao \\ \hline
Export chuẩn & GIFT, QTI, Moodle XML & Trung bình \\ \hline
Ngân hàng câu hỏi & Lưu trữ và tái sử dụng & Thấp \\ \hline
\end{tabular}
\end{table}

\subsubsection{Cải thiện giao diện người dùng}

\begin{table}[H]
\centering
\caption{Kế hoạch cải thiện giao diện}
\label{tab:ui_enhancement}
\begin{tabular}{|p{4cm}|p{6cm}|c|}
\hline
\textbf{Công việc} & \textbf{Mô tả} & \textbf{Độ ưu tiên} \\ \hline
Responsive design & Tối ưu cho mobile và tablet & Cao \\ \hline
Dark mode & Hỗ trợ chế độ tối & Thấp \\ \hline
Keyboard shortcuts & Phím tắt cho các thao tác phổ biến & Trung bình \\ \hline
Progress indicators & Hiển thị tiến độ xử lý chi tiết & Cao \\ \hline
\end{tabular}
\end{table}

\subsection{Hướng phát triển mới}

\subsubsection{Ngắn hạn (6 tháng)}

\begin{enumerate}
\item \textbf{Hệ thống xác thực và phân quyền:}
\begin{itemize}
\item Đăng nhập/đăng ký với email hoặc SSO
\item Phân quyền theo role (Admin, Teacher, Student)
\item Audit log cho các hoạt động quan trọng
\end{itemize}

\item \textbf{API Gateway và Rate Limiting:}
\begin{itemize}
\item Giới hạn số request theo user/API key
\item Caching layer với Redis
\item API versioning
\end{itemize}

\item \textbf{Cải thiện xử lý bất đồng bộ:}
\begin{itemize}
\item Background task queue với Celery
\item WebSocket cho real-time progress
\item Notification system
\end{itemize}
\end{enumerate}

\subsubsection{Trung hạn (1 năm)}

\begin{enumerate}
\item \textbf{Fine-tuning LLM cho giáo dục Việt Nam:}
\begin{itemize}
\item Thu thập dataset giáo dục tiếng Việt
\item Fine-tune Qwen2.5 hoặc Llama3 với LoRA
\item Đánh giá và benchmark với VLUE
\end{itemize}

\item \textbf{Hỗ trợ đa phương thức (Multi-modal):}
\begin{itemize}
\item Phân tích hình ảnh và biểu đồ trong tài liệu
\item Hỗ trợ video lecture transcription
\item Voice-to-text cho input
\end{itemize}

\item \textbf{Tích hợp LMS (Learning Management System):}
\begin{itemize}
\item Plugin cho Moodle
\item Integration với Google Classroom
\item LTI (Learning Tools Interoperability) compliance
\end{itemize}

\item \textbf{Tính năng cộng tác:}
\begin{itemize}
\item Real-time collaborative editing
\item Comments và annotation
\item Version control cho tài liệu
\end{itemize}
\end{enumerate}

\subsubsection{Dài hạn (2+ năm)}

\begin{enumerate}
\item \textbf{Custom Model Training Pipeline:}
\begin{itemize}
\item Interface để người dùng upload training data
\item Automated fine-tuning workflow
\item Model evaluation và deployment
\end{itemize}

\item \textbf{Phân tích học tập (Learning Analytics):}
\begin{itemize}
\item Thống kê kết quả quiz theo chủ đề
\item Phát hiện điểm yếu của học sinh
\item Đề xuất learning path cá nhân hóa
\end{itemize}

\item \textbf{Voice Interaction:}
\begin{itemize}
\item Voice command cho điều khiển ứng dụng
\item Text-to-speech cho output
\item Voice-based Q\&A
\end{itemize}

\item \textbf{Multi-tenant SaaS Deployment:}
\begin{itemize}
\item Kiến trúc cho nhiều tổ chức
\item Isolated data storage per tenant
\item Subscription và billing system
\end{itemize}
\end{enumerate}

\subsection{Roadmap tổng thể}

\begin{figure}[H]
\centering
\begin{verbatim}
Q1 2026: Authentication + API Gateway + Async Processing
    │
Q2 2026: Mobile UI + New Templates + Export Formats
    │
Q3 2026: Multi-modal Support + LMS Integration (Phase 1)
    │
Q4 2026: Fine-tuned Vietnamese Model + Collaborative Features
    │
2027+: Learning Analytics + Voice Interaction + SaaS Platform
\end{verbatim}
\caption{Lộ trình phát triển hệ thống}
\label{fig:roadmap}
\end{figure}

\subsection{Định hướng nghiên cứu}

Ngoài các công việc phát triển sản phẩm, một số hướng nghiên cứu tiềm năng bao gồm:

\begin{enumerate}
\item \textbf{Tối ưu RAG cho tiếng Việt:}
\begin{itemize}
\item Nghiên cứu chunking strategies phù hợp với đặc thù tiếng Việt
\item So sánh các embedding models cho tiếng Việt
\item Hybrid search (semantic + keyword) cho Vietnamese corpus
\end{itemize}

\item \textbf{Evaluation metrics cho AI giáo dục:}
\begin{itemize}
\item Xây dựng benchmark đánh giá chất lượng slide tự động
\item Metrics cho độ khó và chất lượng câu hỏi trắc nghiệm
\item Human evaluation framework
\end{itemize}

\item \textbf{Efficient LLM Inference:}
\begin{itemize}
\item Quantization techniques cho deployment trên edge devices
\item Speculative decoding để tăng tốc inference
\item Model distillation cho lightweight deployment
\end{itemize}
\end{enumerate}

\subsection{Lời kết}

Đồ án ``Hệ thống Phân tích Tài liệu và Tạo Slide Thông minh - AI NVCB'' đã đạt được mục tiêu đề ra: xây dựng một nền tảng AI on-premise phục vụ giáo dục với các chức năng phân tích tài liệu, tạo slide tự động và sinh câu hỏi trắc nghiệm.

Những đóng góp chính của đồ án bao gồm:
\begin{itemize}
\item Giải pháp AI hoàn toàn on-premise đảm bảo bảo mật dữ liệu
\item RAG pipeline tối ưu cho tiếng Việt
\item Kiến trúc linh hoạt cho phép hot-swap LLM models
\item Bộ giải pháp xử lý ngôn ngữ Việt với LLM đa ngôn ngữ
\end{itemize}

Với nền tảng đã xây dựng, hệ thống có tiềm năng phát triển thành một công cụ hỗ trợ giáo dục toàn diện, đặc biệt phù hợp với bối cảnh Việt Nam nơi yêu cầu về bảo mật dữ liệu và chi phí là những yếu tố quan trọng.

\vspace{1em}
\noindent\textit{Ngày hoàn thành: Tháng 12/2025}

\end{document}
