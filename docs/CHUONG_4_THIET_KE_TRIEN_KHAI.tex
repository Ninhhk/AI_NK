\documentclass[../DoAn.tex]{subfiles}
\begin{document}

\setcounter{figure}{9}

\noindent\textbf{Mở đầu chương.}

Chương này trình bày chi tiết quá trình thiết kế, triển khai và đánh giá hệ thống AI NVCB. Nội dung chương bao gồm: (1) Thiết kế kiến trúc phần mềm với biểu đồ gói UML, (2) Thiết kế chi tiết giao diện, lớp và cơ sở dữ liệu, (3) Xây dựng ứng dụng với các công cụ và thư viện cụ thể, (4) Kiểm thử hệ thống, và (5) Triển khai và đánh giá kết quả. Các thiết kế trong chương này được xây dựng dựa trên yêu cầu chức năng (UC01-UC04) và phi chức năng (NFR01-NFR21) đã xác định tại Chương 2, sử dụng các công nghệ đã lựa chọn tại Chương 3.

\section{Thiết kế kiến trúc}

\subsection{Lựa chọn kiến trúc phần mềm}

\subsubsection{Kiến trúc phân lớp (Layered Architecture)}

Hệ thống AI NVCB được xây dựng theo \textbf{kiến trúc phân lớp (Layered Architecture)} kết hợp với \textbf{mô hình Repository Pattern}. Đây là kiến trúc phổ biến trong các ứng dụng enterprise, phân tách hệ thống thành các tầng độc lập với trách nhiệm rõ ràng.

\textbf{Định nghĩa kiến trúc phân lớp:}

Kiến trúc phân lớp tổ chức hệ thống thành các tầng (layers) xếp chồng lên nhau, trong đó mỗi tầng chỉ giao tiếp với tầng liền kề. Nguyên tắc cơ bản là: (i) tầng trên phụ thuộc vào tầng dưới, (ii) tầng dưới không biết đến sự tồn tại của tầng trên, và (iii) không có phụ thuộc bỏ qua tầng (skip-layer dependency).

\textbf{Lý do lựa chọn:}

Thứ nhất, kiến trúc phân lớp đáp ứng yêu cầu NFR18 về tính dễ bảo trì đã nêu tại mục 2.4.5. Việc tách biệt các tầng cho phép thay đổi implementation ở một tầng mà không ảnh hưởng đến các tầng khác.

Thứ hai, kiến trúc này phù hợp với stack công nghệ đã chọn tại Chương 3: FastAPI (backend API), Streamlit (frontend), và LangChain/Ollama (AI services). Mỗi công nghệ được ánh xạ vào một tầng cụ thể.

Thứ ba, Repository Pattern được bổ sung để tách biệt logic truy cập dữ liệu, giúp dễ dàng thay đổi từ SQLite sang PostgreSQL hoặc database khác khi cần mở rộng quy mô.

\subsubsection{Áp dụng kiến trúc vào hệ thống AI NVCB}

Hệ thống AI NVCB được tổ chức thành \textbf{4 tầng chính}:

\begin{table}[H]
\centering
    \begin{tabular}{|p{3.3cm}|p{3.3cm}|p{3.3cm}|p{3.3cm}|}
        \hline
        \textbf{Tầng} & \textbf{Tên gọi} & \textbf{Thành phần trong hệ thống} & \textbf{Trách nhiệm} \\ \\ \hline
        \textbf{Tầng 1} & Presentation Layer & frontend/ (Streamlit) & Giao diện người dùng, xử lý input/output \\ \\ \hline
        \textbf{Tầng 2} & API Layer & backend/api/ (FastAPI) & Định nghĩa REST API endpoints, validation \\ \\ \hline
        \textbf{Tầng 3} & Business Logic Layer & backend/document\_analysis/, backend/slide\_generation/, backend/model\_management/ & Xử lý nghiệp vụ chính: phân tích tài liệu, tạo slide, quản lý model \\ \\ \hline
        \textbf{Tầng 4} & Data Access Layer & utils/repository.py, utils/database.py & Truy cập CSDL, lưu trữ file \\ \\ \hline
    \end{tabular}
    \caption{Áp dụng kiến trúc vào hệ thống AI NVCB}
    \label{tab:chuong4_1}
\end{table}

\textbf{Thành phần bổ sung:}
\begin{itemize}
    \item \textbf{External Services}: Ollama LLM Server, FAISS Vector Store - các dịch vụ bên ngoài mà hệ thống tích hợp
    \item \textbf{Utilities}: utils/ - các module tiện ích dùng chung (logging, cleanup, health check)
\end{itemize}

\subsubsection{So sánh với kiến trúc MVC truyền thống}

\begin{table}[H]
\centering
    \begin{tabular}{|p{3.3cm}|p{3.3cm}|p{3.3cm}|}
        \hline
        \textbf{Thành phần MVC} & \textbf{Tương ứng trong AI NVCB} & \textbf{Mở rộng/Cải tiến} \\ \\ \hline
        \textbf{Model} & utils/repository.py + Pydantic Models & Sử dụng Repository Pattern thay vì Active Record \\ \\ \hline
        \textbf{View} & frontend/ (Streamlit pages) & Tách biệt hoàn toàn khỏi backend qua REST API \\ \\ \hline
        \textbf{Controller} & backend/api/ (FastAPI routers) & Tách thành API Layer + Service Layer \\ \\ \hline
    \end{tabular}
    \caption{So sánh với kiến trúc MVC truyền thống}
    \label{tab:chuong4_2}
\end{table}

Điểm cải tiến so với MVC truyền thống là việc tách Controller thành hai phần: API Layer (chỉ xử lý HTTP request/response) và Service Layer (chứa business logic). Điều này tuân theo nguyên tắc Single Responsibility Principle (SRP).

\subsection{Thiết kế tổng quan}

\subsubsection{Biểu đồ gói UML (Package Diagram)}

Biểu đồ dưới đây thể hiện cấu trúc các gói trong hệ thống AI NVCB và mối quan hệ phụ thuộc giữa chúng. Các gói được sắp xếp theo tầng từ trên xuống dưới.

\begin{figure}[H]
    \centering
    \includegraphics{Hinhve/Picture10.png}
    \caption{Ví dụ biểu đồ phụ thuộc gói}
    \label{fig:Fig10}
\end{figure}

\subsubsection{Mô tả chi tiết các gói}

\textbf{Tầng Trình Bày (Presentation Layer):}

\begin{table}[H]
\centering
    \begin{tabular}{|p{3.3cm}|p{3.3cm}|p{3.3cm}|}
        \hline
        \textbf{Gói} & \textbf{Mô tả} & \textbf{Liên kết Use Case} \\ \\ \hline
        frontend.pages & Các trang giao diện chính của ứng dụng Streamlit & UC01-UC04 \\ \\ \hline
        frontend.components & Các component tái sử dụng (sidebar, chat, system prompt) & Hỗ trợ UC01.4 \\ \\ \hline
        frontend.app & Entry point của ứng dụng frontend & - \\ \\ \hline
    \end{tabular}
    \caption{Mô tả chi tiết các gói}
    \label{tab:chuong4_3}
\end{table}

\textbf{Tầng API (API Layer):}

\begin{table}[H]
\centering
    \begin{tabular}{|p{3.3cm}|p{3.3cm}|p{3.3cm}|}
        \hline
        \textbf{Gói} & \textbf{Mô tả} & \textbf{Endpoints chính} \\ \\ \hline
        backend.api & FastAPI routers định nghĩa REST API & /api/documents/*, /api/slides/*, /api/ollama/* \\ \\ \hline
    \end{tabular}
    \caption{Mô tả chi tiết các gói}
    \label{tab:chuong4_4}
\end{table}

Chi tiết các router:
\begin{itemize}
    \item document\_routes.py: Upload tài liệu, phân tích, Q\&A, tạo quiz (UC01, UC03)
    \item slide\_routes.py: Tạo slide từ chủ đề/tài liệu (UC02)
    \item model\_routes.py: Quản lý model Ollama (UC04)
    \item health\_routes.py: Health check endpoints (NFR06)
    \item cleanup\_routes.py: Dọn dẹp storage (NFR09)
\end{itemize}

\textbf{Tầng Nghiệp Vụ (Business Logic Layer):}

\begin{table}[H]
\centering
    \begin{tabular}{|p{0.25\textwidth}|p{0.30\textwidth}|p{0.38\textwidth}|}
        \hline
        \textbf{Gói} & \textbf{Mô tả} & \textbf{Lớp chính} \\ \hline
        backend.document\_analysis & Xử lý phân tích tài liệu và RAG & DocumentAnalysisService \\ \hline
        backend.slide\_generation & Tạo nội dung slide và file PPTX & SlideGenerationService, PowerPointGenerator \\ \hline
        backend.model\_management & Quản lý model LLM và system prompt & ModelManager, SystemPromptManager \\ \hline
    \end{tabular}
    \caption{Mô tả chi tiết các gói}
    \label{tab:chuong4_5}
\end{table}

\textbf{Tầng Truy Cập Dữ Liệu (Data Access Layer):}

\begin{table}[H]
\centering
    \begin{tabular}{|p{3.3cm}|p{3.3cm}|p{3.3cm}|}
        \hline
        \textbf{Gói} & \textbf{Mô tả} & \textbf{Pattern áp dụng} \\ \\ \hline
        utils.repository & CRUD operations cho documents, chat history & Repository Pattern \\ \\ \hline
        utils.database & Kết nối SQLite, quản lý schema & Context Manager \\ \\ \hline
    \end{tabular}
    \caption{Mô tả chi tiết các gói}
    \label{tab:chuong4_6}
\end{table}

\subsection{Thiết kế chi tiết gói}

\subsubsection{4.1.3.1 Gói backend.api}

Gói này chứa các FastAPI routers, đóng vai trò Controller trong kiến trúc. Mỗi router xử lý một nhóm chức năng cụ thể.

\begin{figure}[H]
    \centering
    \includegraphics{Hinhve/Picture11.png}
    \caption{Ví dụ biểu đồ phụ thuộc gói}
    \label{fig:Fig11}
\end{figure}

\textbf{Giải thích thiết kế:}

Mỗi router được thiết kế theo nguyên tắc Single Responsibility - chỉ xử lý một nhóm chức năng liên quan. Router không chứa business logic, chỉ thực hiện: (i) validate input với Pydantic, (ii) gọi service tương ứng, và (iii) format response.

\subsubsection{4.1.3.2 Gói backend.document\_analysis}

Gói này chứa service xử lý nghiệp vụ phân tích tài liệu, triển khai kỹ thuật RAG đã trình bày tại mục 3.5.2.

\begin{figure}[H]
    \centering
    \includegraphics[width=0.9\linewidth]{Hinhve/Picture12.png}
    \caption{Ví dụ biểu đồ phụ thuộc gói}
    \label{fig:Fig12}
\end{figure}

\textbf{Giải thích thiết kế:}

DocumentAnalysisService là lớp chính của gói, sử dụng Composition để tích hợp các thành phần LangChain (Ollama LLM, HuggingFace Embeddings). Service này triển khai các use case UC01.1-UC01.4 và UC03.

Mối quan hệ với GlobalModelConfig và SystemPromptManager cho phép service đồng bộ cấu hình model trên toàn hệ thống, đáp ứng yêu cầu UC04.3.

\subsubsection{4.1.3.3 Gói backend.slide\_generation}

\begin{figure}[H]
    \centering
    \includegraphics[width=0.75\linewidth]{Hinhve/Picture13.png}
    \caption{Ví dụ biểu đồ phụ thuộc gói}
    \label{fig:Fig13}
\end{figure}

\textbf{Giải thích thiết kế:}

SlideGenerationService orchestrates toàn bộ quy trình tạo slide (UC02), từ việc parse tài liệu đầu vào đến việc gọi LLM sinh nội dung. PowerPointGenerator được tách riêng theo nguyên tắc SRP, chỉ chịu trách nhiệm tạo file PPTX từ dữ liệu JSON.

\subsubsection{4.1.3.4 Gói backend.model\_management}

\begin{figure}[H]
    \centering
    \includegraphics{Hinhve/Picture14.png}
    \caption{Ví dụ biểu đồ phụ thuộc gói}
    \label{fig:Fig14}
\end{figure}

\textbf{Giải thích thiết kế:}

GlobalModelConfig và SystemPromptManager sử dụng \textbf{Singleton Pattern} để đảm bảo chỉ có một instance duy nhất trên toàn hệ thống. Điều này quan trọng vì cấu hình model cần được đồng bộ giữa các service (DocumentAnalysisService và SlideGenerationService).

ModelManager sử dụng async/await để xử lý việc download model trong background, tránh block request của người dùng.

\subsubsection{4.1.3.5 Gói utils (Data Access Layer)}

\begin{figure}[H]
    \centering
    \includegraphics[width=1\linewidth]{Hinhve/Picture15.png}
    \caption{Ví dụ biểu đồ phụ thuộc gói}
    \label{fig:Fig15}
\end{figure}

\textbf{Giải thích thiết kế:}

Gói utils triển khai \textbf{Repository Pattern} để tách biệt logic truy cập dữ liệu khỏi business logic. Mỗi entity (Document, ChatHistory, Slide) có repository riêng với các phương thức CRUD chuẩn.

DatabaseConnection sử dụng \textbf{Context Manager Pattern} để đảm bảo connection được đóng đúng cách, tránh resource leak.

\subsection{Tổng kết thiết kế kiến trúc}

Kiến trúc phân lớp của hệ thống AI NVCB đảm bảo các nguyên tắc thiết kế sau:

\begin{table}[H]
\centering
    \begin{tabular}{|p{4.5cm}|p{8cm}|}
        \hline
        \textbf{Nguyên tắc} & \textbf{Cách áp dụng} \\ \hline
        \textbf{Separation of Concerns} & Mỗi tầng có trách nhiệm riêng biệt \\ \hline
        \textbf{Single Responsibility} & Mỗi class/module thực hiện một nhiệm vụ cụ thể \\ \hline
        \textbf{Dependency Inversion} & Business layer không phụ thuộc trực tiếp vào database, mà qua Repository \\ \hline
        \textbf{Don't Repeat Yourself} & GlobalModelConfig và SystemPromptManager dùng chung cho các service \\ \hline
    \end{tabular}
    \caption{Tổng kết thiết kế kiến trúc}
    \label{tab:chuong4_7}
\end{table}

Kiến trúc này đáp ứng các yêu cầu phi chức năng:
\begin{itemize}
    \item \textbf{NFR18}: Tính dễ bảo trì - có thể thay đổi một tầng mà không ảnh hưởng tầng khác
    \item \textbf{NFR21}: Docker containerization - mỗi tầng có thể được đóng gói riêng
    \item \textbf{NFR14}: Bảo mật - dữ liệu chỉ được truy cập qua Repository layer
\end{itemize}

\section{Thiết kế chi tiết}

\subsection{Thiết kế giao diện}

\subsubsection{Đặc tả môi trường hiển thị}

Hệ thống AI NVCB được thiết kế như ứng dụng web responsive, tối ưu cho trải nghiệm trên desktop. Đặc tả kỹ thuật giao diện như sau:

\begin{table}[H]
\centering
\begin{tabular}{|p{0.25\textwidth}|p{0.25\textwidth}|p{0.43\textwidth}|}
\hline
\textbf{Thuộc tính} & \textbf{Giá trị} & \textbf{Ghi chú} \\ \hline
\textbf{Độ phân giải tối thiểu} & 1280 x 720 px & HD ready \\ \hline
\textbf{Độ phân giải khuyến nghị} & 1920 x 1080 px & Full HD \\ \hline
\textbf{Layout} & Wide mode & Tận dụng tối đa không gian màn hình \\ \hline
\textbf{Responsive} & Desktop-first & Hỗ trợ tablet, không tối ưu mobile \\ \hline
\textbf{Color depth} & 24-bit (True Color) & 16.7 triệu màu \\ \hline
\end{tabular}
\caption{Đặc tả môi trường hiển thị của ứng dụng}
\label{tab:chuong4_8}
\end{table}

\subsubsection{Quy chuẩn thiết kế giao diện}

\textbf{1. Bảng màu (Color Palette):}

Hệ thống sử dụng Dark Theme để giảm mỏi mắt khi làm việc lâu, phù hợp với đối tượng người dùng là giáo viên và nhân viên văn phòng (mục 2.1.1).

\begin{table}[H]
\centering
    \begin{tabular}{|p{3.3cm}|p{3.3cm}|p{3.3cm}|}
        \hline
        \textbf{Biến CSS} & \textbf{Giá trị Hex} & \textbf{Mục đích sử dụng} \\ \\ \hline
        --primary-color & \#1f77b4 & Màu chính, buttons, links \\ \\ \hline
        --secondary-color & \#2c3e50 & Màu phụ, headers \\ \\ \hline
        --accent-color & \#3498db & Điểm nhấn, hover states \\ \\ \hline
        --background-dark & \#0E1117 & Nền chính \\ \\ \hline
        --card-background & \#262730 & Nền thẻ, containers \\ \\ \hline
        --text-primary & \#FAFAFA & Văn bản chính \\ \\ \hline
        --text-secondary & \#888888 & Văn bản phụ \\ \\ \hline
        --success-color & \#2ecc71 & Trạng thái thành công \\ \\ \hline
        --error-color & \#e74c3c & Trạng thái lỗi \\ \\ \hline
        --warning-color & \#f1c40f & Trạng thái cảnh báo \\ \\ \hline
    \end{tabular}
    \caption{Quy chuẩn thiết kế giao diện}
    \label{tab:chuong4_9}
\end{table}

\textbf{2. Quy chuẩn thành phần giao diện:}

\begin{table}[H]
\centering
    \begin{tabular}{|p{3.3cm}|p{3.3cm}|}
        \hline
        \textbf{Thành phần} & \textbf{Quy chuẩn} \\ \\ \hline
        \textbf{Buttons} & Border-radius: 8px, Height: 3em, Gradient background \\ \\ \hline
        \textbf{Input fields} & Border-radius: 8px, Border: 1px solid rgba(255,255,255,0.1) \\ \\ \hline
        \textbf{Cards} & Border-radius: 10px, Padding: 1.5rem, Box-shadow: 0 4px 6px \\ \\ \hline
        \textbf{File uploader} & Border: 2px dashed, Background: card-background \\ \\ \hline
        \textbf{Headers (h1-h4)} & Font-weight: 600, Color: text-primary \\ \\ \hline
    \end{tabular}
    \caption{Quy chuẩn thiết kế giao diện}
    \label{tab:chuong4_10}
\end{table}

\textbf{3. Quy chuẩn hiển thị trạng thái:}

\begin{table}[H]
\centering
    \begin{tabular}{|p{3.3cm}|p{3.3cm}|p{3.3cm}|}
        \hline
        \textbf{Trạng thái} & \textbf{Hiển thị} & \textbf{Component Streamlit} \\ \\ \hline
        Đang xử lý & Spinner + thông báo & st.spinner() \\ \\ \hline
        Thành công & Toast màu xanh lá & st.success() \\ \\ \hline
        Lỗi & Toast màu đỏ & st.error() \\ \\ \hline
        Cảnh báo & Toast màu vàng & st.warning() \\ \\ \hline
        Thông tin & Toast màu xanh dương & st.info() \\ \\ \hline
    \end{tabular}
    \caption{Quy chuẩn thiết kế giao diện}
    \label{tab:chuong4_11}
\end{table}

Quy chuẩn này đáp ứng yêu cầu NFR12 (hiển thị trạng thái realtime) và NFR13 (thông báo lỗi rõ ràng).

\subsubsection{Thiết kế mockup các màn hình chính}

\textbf{Màn hình 1: Trang chủ (Home)}
\begin{figure}
    \centering
    \includegraphics[width=0.75\linewidth]{Hinhve/Picture16.png}
    \caption{Trang chủ}
    \label{fig:Fig16}
\end{figure}

\textbf{Màn hình 2: Phân tích Tài liệu (UC01)}
\begin{figure}
    \centering
    \includegraphics[width=0.75\linewidth]{Hinhve/Picture17.png}
    \caption{Phân tích tài liệu}
    \label{fig:Fig17}
\end{figure}

\textbf{Màn hình 3: Tạo Slide AI (UC02)}
\begin{figure}
    \centering
    \includegraphics[width=0.75\linewidth]{Hinhve/Picture18.png}
    \caption{Tạo slide AI}
    \label{fig:Fig18}
\end{figure}

\textbf{Màn hình 4: Quản lý Model AI (UC04)}
\begin{figure}
    \centering
    \includegraphics[width=0.75\linewidth]{Hinhve/Picture19.png}
    \caption{Quản lý Model AI}
    \label{fig:Fig19}
\end{figure}

\subsection{Thiết kế lớp}

\subsubsection{Thiết kế chi tiết các lớp chủ đạo}

\textbf{Lớp 1: DocumentAnalysisService}

Lớp này là core service xử lý nghiệp vụ phân tích tài liệu, triển khai các use case UC01.1-UC01.4 và UC03.

\begin{table}[H]
\centering
    \begin{tabular}{|p{3.3cm}|p{3.3cm}|p{3.3cm}|}
        \hline
        \textbf{Thuộc tính} & \textbf{Kiểu dữ liệu} & \textbf{Mô tả} \\ \\ \hline
        model\_name & str & Tên model LLM đang sử dụng \\ \\ \hline
        temperature & float & Tham số temperature cho LLM \\ \\ \hline
        base\_url & str & URL của Ollama server \\ \\ \hline
        llm & Ollama & Instance của LangChain Ollama wrapper \\ \\ \hline
        embeddings & HuggingFaceEmbeddings & Model tạo embeddings \\ \\ \hline
        chat\_histories & Dict[str, List] & Lưu trữ lịch sử hội thoại theo document\_id \\ \\ \hline
    \end{tabular}
    \caption{Thiết kế chi tiết các lớp chủ đạo}
    \label{tab:chuong4_12}
\end{table}

\begin{table}[H]
\centering
    \begin{tabular}{|p{3.3cm}|p{3.3cm}|p{3.3cm}|p{3.3cm}|}
        \hline
        \textbf{Phương thức} & \textbf{Tham số} & \textbf{Trả về} & \textbf{Mô tả} \\ \\ \hline
        \_\_init\_\_ & model\_name, base\_url, temperature & None & Khởi tạo service với cấu hình \\ \\ \hline
        analyze\_document & file\_content, query\_type, user\_query, system\_prompt & Dict[str, str] & Phân tích tài liệu đơn \\ \\ \hline
        analyze\_multiple\_documents & file\_contents, filenames, query, system\_prompt & Dict & Phân tích nhiều tài liệu với RAG \\ \\ \hline
        generate\_quiz & file\_content, num\_questions, difficulty & Dict[str, Any] & Tạo quiz từ tài liệu đơn \\ \\ \hline
        generate\_quiz\_multiple & file\_contents, filenames, num\_questions, difficulty & Dict & Tạo quiz từ nhiều tài liệu \\ \\ \hline
        set\_model & model\_name & None & Thay đổi model LLM \\ \\ \hline
        get\_current\_model & - & str & Lấy tên model hiện tại \\ \\ \hline
        \_load\_document & file\_path, start\_page, end\_page & List[Document] & Load và parse tài liệu \\ \\ \hline
        \_initialize\_model & - & Ollama & Khởi tạo LLM instance \\ \\ \hline
        add\_to\_chat\_history & document\_id, user\_query, system\_response & None & Lưu lịch sử chat \\ \\ \hline
    \end{tabular}
    \caption{Thiết kế chi tiết các lớp chủ đạo}
    \label{tab:chuong4_13}
\end{table}

\textbf{Lớp 2: SlideGenerationService}

Lớp này xử lý nghiệp vụ tạo slide, triển khai use case UC02.

\begin{table}[H]
\centering
    \begin{tabular}{|p{3.3cm}|p{3.3cm}|p{3.3cm}|}
        \hline
        \textbf{Thuộc tính} & \textbf{Kiểu dữ liệu} & \textbf{Mô tả} \\ \\ \hline
        model\_name & str & Tên model LLM \\ \\ \hline
        base\_url & str & URL Ollama server \\ \\ \hline
        llm & Ollama & Instance LangChain Ollama \\ \\ \hline
        pptx\_generator & PowerPointGenerator & Generator tạo file PPTX \\ \\ \hline
    \end{tabular}
    \caption{Thiết kế chi tiết các lớp chủ đạo}
    \label{tab:chuong4_14}
\end{table}

\begin{table}[H]
\centering
    \begin{tabular}{|p{3.3cm}|p{3.3cm}|p{3.3cm}|p{3.3cm}|}
        \hline
        \textbf{Phương thức} & \textbf{Tham số} & \textbf{Trả về} & \textbf{Mô tả} \\ \\ \hline
        \_\_init\_\_ & model\_name, base\_url & None & Khởi tạo service \\ \\ \hline
        generate\_slides & topic, num\_slides, document\_content, system\_prompt & Dict[str, List] & Tạo slides từ chủ đề \\ \\ \hline
        parse\_document & file\_content, file\_type & str & Parse tài liệu sang text \\ \\ \hline
        set\_model & model\_name & None & Thay đổi model \\ \\ \hline
        get\_current\_model & - & str & Lấy model hiện tại \\ \\ \hline
        \_parse\_pdf & file\_content & str & Parse file PDF \\ \\ \hline
        \_parse\_docx & file\_content & str & Parse file DOCX \\ \\ \hline
        \_parse\_txt & file\_content & str & Parse file TXT \\ \\ \hline
        \_save\_slides & topic, slides\_data & str & Lưu slides JSON \\ \\ \hline
        \_invoke\_model & prompt, system\_prompt & str & Gọi LLM sinh nội dung \\ \\ \hline
    \end{tabular}
    \caption{Thiết kế chi tiết các lớp chủ đạo}
    \label{tab:chuong4_15}
\end{table}

\textbf{Lớp 3: PowerPointGenerator}

Lớp này tạo file PPTX từ dữ liệu JSON, sử dụng thư viện python-pptx.

\begin{table}[H]
\centering
    \begin{tabular}{|p{3.3cm}|p{3.3cm}|p{3.3cm}|}
        \hline
        \textbf{Thuộc tính} & \textbf{Kiểu dữ liệu} & \textbf{Mô tả} \\ \\ \hline
        prs & Presentation & Instance của python-pptx Presentation \\ \\ \hline
    \end{tabular}
    \caption{Thiết kế chi tiết các lớp chủ đạo}
    \label{tab:chuong4_16}
\end{table}

\begin{table}[H]
\centering
    \begin{tabular}{|p{3.3cm}|p{3.3cm}|p{3.3cm}|p{3.3cm}|}
        \hline
        \textbf{Phương thức} & \textbf{Tham số} & \textbf{Trả về} & \textbf{Mô tả} \\ \\ \hline
        \_\_init\_\_ & - & None & Khởi tạo Presentation rỗng \\ \\ \hline
        generate\_presentation & slides\_data, output\_path & str & Tạo file PPTX hoàn chỉnh \\ \\ \hline
        add\_slide & slide\_data & None & Thêm một slide với title và bullets \\ \\ \hline
    \end{tabular}
    \caption{Thiết kế chi tiết các lớp chủ đạo}
    \label{tab:chuong4_17}
\end{table}

\textbf{Lớp 4: DocumentRepository}

Lớp này triển khai Repository Pattern cho entity Document, tách biệt logic CRUD khỏi business logic.

\begin{table}[H]
\centering
    \begin{tabular}{|p{3.3cm}|p{3.3cm}|p{3.3cm}|p{3.3cm}|}
        \hline
        \textbf{Phương thức} & \textbf{Tham số} & \textbf{Trả về} & \textbf{Mô tả} \\ \\ \hline
        insert\_document & user\_id, filename, path, content\_type, size, meta & Dict[str, Any] & Thêm document mới \\ \\ \hline
        get\_document\_by\_id & document\_id & Optional[Dict] & Lấy document theo ID \\ \\ \hline
        get\_documents\_by\_user & user\_id, limit, offset & List[Dict] & Lấy danh sách document của user \\ \\ \hline
        update\_document\_content & document\_id, content & bool & Cập nhật nội dung document \\ \\ \hline
        update\_document\_meta & document\_id, meta & bool & Cập nhật metadata \\ \\ \hline
        delete\_document & document\_id & bool & Xóa document \\ \\ \hline
        get\_old\_documents & cutoff\_date & List[Dict] & Lấy documents cũ hơn cutoff \\ \\ \hline
        insert\_or\_get\_document & document\_id, user\_id, filename, ... & Dict & Insert hoặc trả về document có sẵn \\ \\ \hline
    \end{tabular}
    \caption{Thiết kế chi tiết các lớp chủ đạo}
    \label{tab:chuong4_18}
\end{table}

\subsubsection{Biểu đồ trình tự cho các Use Case quan trọng}

\textbf{Biểu đồ trình tự UC01.3: Hỏi đáp tài liệu (RAG Q\&A)}

\begin{figure}[H]
    \centering
    \includegraphics{Hinhve/Picture16.png}
    \caption{Ví dụ biểu đồ phụ thuộc gói}
    \label{fig:SeqUC01_3}
\end{figure}

\textbf{Giải thích luồng xử lý:}

1. Người dùng tải file và nhập câu hỏi trên giao diện Streamlit
2. Frontend gửi request POST đến API endpoint /api/documents/analyze
3-4. API kiểm tra và lưu document vào database (tránh duplicate)
5. API gọi DocumentAnalysisService.analyze\_document() với query\_type="qa"
6-7. Service load và chia nhỏ văn bản thành chunks (1000 ký tự, overlap 200)
8-9. Tạo FAISS vector store từ chunks với HuggingFace embeddings
10-11. Tìm kiếm 3 chunks liên quan nhất với câu hỏi (similarity search)
12. Tạo prompt kết hợp context và câu hỏi
13-14. Gọi Ollama LLM để sinh câu trả lời
15. Lưu vào chat history trong memory
16-18. Trả về kết quả và lưu vào database
19-21. Hiển thị kết quả cho người dùng

\textbf{Biểu đồ trình tự UC02: Tạo Slide AI}

\begin{figure}[H]
    \centering
    \includegraphics{Hinhve/Picture17.png}
    \caption{Ví dụ biểu đồ phụ thuộc gói}
    \label{fig:SeqUC02}
\end{figure}

\textbf{Giải thích luồng xử lý:}

1. Người dùng nhập chủ đề, chọn số slide (1-20), có thể upload tài liệu tham khảo
2. Frontend gửi request với multipart/form-data
3a-3c. Nếu có tài liệu, API gọi service parse nội dung text
4. API gọi generate\_slides() với các tham số
5. Service tạo prompt từ template, thêm context từ tài liệu
6-10. Gọi LLM và retry tối đa 3 lần nếu JSON không hợp lệ
11. Validate và normalize dữ liệu slides
12-15. PowerPointGenerator tạo file PPTX với python-pptx
17-18. Lưu JSON backup để có thể chỉnh sửa sau
19-21. Trả về kết quả và hiển thị preview
22-27. Người dùng tải xuống file PPTX

\subsection{Thiết kế cơ sở dữ liệu}

\subsubsection{Biểu đồ thực thể liên kết (ER Diagram)}

\begin{figure}[H]
    \centering
    \includegraphics{Hinhve/Picture18.png}
    \caption{Ví dụ biểu đồ phụ thuộc gói}
    \label{fig:ERDiagram}
\end{figure}

\subsubsection{Mô tả chi tiết các bảng}

\textbf{Bảng 1: documents}

Lưu trữ thông tin tài liệu đã upload, phục vụ UC01.1.

\begin{table}[H]
\centering
    \begin{tabular}{|p{3.3cm}|p{3.3cm}|p{3.3cm}|p{3.3cm}|}
        \hline
        \textbf{Cột} & \textbf{Kiểu} & \textbf{Ràng buộc} & \textbf{Mô tả} \\ \\ \hline
        id & TEXT & PRIMARY KEY & UUID dựa trên nội dung file (content-based ID) \\ \\ \hline
        user\_id & TEXT & NOT NULL & ID người dùng sở hữu \\ \\ \hline
        filename & TEXT & NOT NULL & Tên file gốc \\ \\ \hline
        path & TEXT & NOT NULL & Đường dẫn file trên storage \\ \\ \hline
        content\_type & TEXT & - & MIME type (application/pdf, ...) \\ \\ \hline
        size & INTEGER & - & Kích thước file (bytes) \\ \\ \hline
        content & TEXT & - & Nội dung text đã extract \\ \\ \hline
        hash & TEXT & - & MD5 hash của file \\ \\ \hline
        meta & TEXT & - & Metadata JSON (pages, author, ...) \\ \\ \hline
        created\_at & INTEGER & NOT NULL & Unix timestamp tạo \\ \\ \hline
        updated\_at & INTEGER & NOT NULL & Unix timestamp cập nhật \\ \\ \hline
    \end{tabular}
    \caption{Mô tả chi tiết các bảng}
    \label{tab:chuong4_19}
\end{table}

\textbf{Bảng 2: chat\_history}

Lưu trữ lịch sử hội thoại Q\&A, phục vụ UC01.4.

\begin{table}[H]
\centering
    \begin{tabular}{|p{3.3cm}|p{3.3cm}|p{3.3cm}|p{3.3cm}|}
        \hline
        \textbf{Cột} & \textbf{Kiểu} & \textbf{Ràng buộc} & \textbf{Mô tả} \\ \\ \hline
        id & TEXT & PRIMARY KEY & UUID ngẫu nhiên \\ \\ \hline
        document\_id & TEXT & FOREIGN KEY & Liên kết đến documents.id \\ \\ \hline
        user\_query & TEXT & NOT NULL & Câu hỏi của người dùng \\ \\ \hline
        system\_response & TEXT & - & Câu trả lời của AI \\ \\ \hline
        meta & TEXT & - & Metadata (model used, latency, ...) \\ \\ \hline
        created\_at & INTEGER & NOT NULL & Unix timestamp \\ \\ \hline
        updated\_at & INTEGER & NOT NULL & Unix timestamp \\ \\ \hline
    \end{tabular}
    \caption{Mô tả chi tiết các bảng}
    \label{tab:chuong4_20}
\end{table}

\textbf{Ràng buộc:} FOREIGN KEY (document\_id) REFERENCES documents(id) ON DELETE CASCADE

\textbf{Bảng 3: slides}

Lưu trữ lịch sử tạo slide, phục vụ UC02.

\begin{table}[H]
\centering
    \begin{tabular}{|p{3.3cm}|p{3.3cm}|p{3.3cm}|p{3.3cm}|}
        \hline
        \textbf{Cột} & \textbf{Kiểu} & \textbf{Ràng buộc} & \textbf{Mô tả} \\ \\ \hline
        id & TEXT & PRIMARY KEY & UUID ngẫu nhiên \\ \\ \hline
        user\_id & TEXT & NOT NULL & ID người tạo \\ \\ \hline
        title & TEXT & NOT NULL & Chủ đề/tiêu đề bài trình bày \\ \\ \hline
        slide\_count & INTEGER & - & Số lượng slides \\ \\ \hline
        content & TEXT & - & Nội dung slides dạng JSON \\ \\ \hline
        json\_path & TEXT & - & Đường dẫn file JSON backup \\ \\ \hline
        pptx\_path & TEXT & - & Đường dẫn file PPTX \\ \\ \hline
        meta & TEXT & - & Metadata (model, documents used, ...) \\ \\ \hline
        created\_at & INTEGER & NOT NULL & Unix timestamp \\ \\ \hline
        updated\_at & INTEGER & NOT NULL & Unix timestamp \\ \\ \hline
    \end{tabular}
    \caption{Mô tả chi tiết các bảng}
    \label{tab:chuong4_21}
\end{table}

\textbf{Bảng 4: quizzes}

Lưu trữ lịch sử tạo quiz, phục vụ UC03.

\begin{table}[H]
\centering
    \begin{tabular}{|p{3.3cm}|p{3.3cm}|p{3.3cm}|p{3.3cm}|}
        \hline
        \textbf{Cột} & \textbf{Kiểu} & \textbf{Ràng buộc} & \textbf{Mô tả} \\ \\ \hline
        id & TEXT & PRIMARY KEY & UUID ngẫu nhiên \\ \\ \hline
        document\_id & TEXT & FOREIGN KEY & Liên kết đến documents.id \\ \\ \hline
        questions\_count & INTEGER & - & Số câu hỏi \\ \\ \hline
        difficulty & TEXT & - & Mức độ khó (easy/medium/hard) \\ \\ \hline
        content & TEXT & - & Nội dung quiz dạng JSON \\ \\ \hline
        meta & TEXT & - & Metadata \\ \\ \hline
        created\_at & INTEGER & NOT NULL & Unix timestamp \\ \\ \hline
        updated\_at & INTEGER & NOT NULL & Unix timestamp \\ \\ \hline
    \end{tabular}
    \caption{Mô tả chi tiết các bảng}
    \label{tab:chuong4_22}
\end{table}

\textbf{Ràng buộc:} FOREIGN KEY (document\_id) REFERENCES documents(id) ON DELETE CASCADE

\subsubsection{Lý do chọn SQLite}

SQLite được chọn làm hệ quản trị CSDL chính vì:

1. \textbf{Embedded database}: Không cần cài đặt server riêng, phục vụ triển khai on-premise đơn giản (NFR14)
2. \textbf{Zero-configuration}: Không cần quản trị, tự động tạo file database
3. \textbf{Đủ cho quy mô dự án}: Hỗ trợ tối thiểu 10 users đồng thời (NFR04)
4. \textbf{Dễ backup}: Chỉ cần copy một file .sqlite
5. \textbf{Tương thích}: Có thể migrate sang PostgreSQL khi cần scale

\subsubsection{Cơ chế CASCADE DELETE}

Khi xóa một document, tất cả chat\_history và quizzes liên quan sẽ tự động bị xóa theo, đảm bảo tính toàn vẹn dữ liệu và tránh orphan records.

\section{Xây dựng ứng dụng}
\subsection{Thư viện và công cụ sử dụng}
Nội dung chi tiết được chuyển sang Phụ lục C.
\subsection{Kết quả đạt được}
Nội dung chi tiết được chuyển sang Phụ lục C.
\subsection{Minh họa các chức năng chính}
Nội dung chi tiết được chuyển sang Phụ lục C.

\section{Kiểm thử}
Nội dung chi tiết được chuyển sang Phụ lục C.

\section{Triển khai}
Nội dung chi tiết được chuyển sang Phụ lục C.

\end{document}