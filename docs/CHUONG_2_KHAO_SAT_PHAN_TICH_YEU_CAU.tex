\documentclass[../DoAn.tex]{subfiles}
\begin{document}

Chương này có độ dài từ 9 đến 11 trang.

Với phương pháp phân tích thiết kế hướng đối tượng, sinh viên sử dụng biểu đồ use case theo hướng dẫn của template này. Với các phương pháp khác, sinh viên trao đổi với giáo viên hướng dẫn để đổi tên và sắp xếp lại đề mục cho phù hợp. Ví dụ, thay vì sử dụng biểu đồ use case, sinh viên đi theo hướng tiếp cận Agile có thể dùng User Story.


    extbf{Lưu ý}: Mỗi chương nên có thêm 1 đoạn mở đầu chương và kết thúc chương, mở đầu giới thiệu những nội dung sẽ trình bày trong chương, kết thúc tổng kết lại các nội dung đã trình bày

\section*{Mở đầu chương}

Chương này trình bày quá trình khảo sát hiện trạng và phân tích yêu cầu cho hệ thống \textbf{AI NVCB - Hệ thống Phân tích Tài liệu và Tạo Slide thông minh}. Nội dung chương bao gồm: (1) Khảo sát hiện trạng các hệ thống tương tự trên thị trường, (2) Tổng quan chức năng thông qua biểu đồ use case, (3) Đặc tả chi tiết các use case quan trọng, và (4) Các yêu cầu phi chức năng của hệ thống. Qua đó, xác định rõ ràng phạm vi và yêu cầu cần phát triển cho hệ thống.

\section{Khảo sát hiện trạng}
\label{section:2.1}

\subsection{Nhu cầu thực tế}
\label{subsection:2.1.1}

Trong bối cảnh chuyển đổi số và ứng dụng trí tuệ nhân tạo (AI) ngày càng phổ biến, nhu cầu về các công cụ hỗ trợ phân tích tài liệu và tạo nội dung tự động ngày càng tăng cao. Các đối tượng sử dụng chính bao gồm:

\begin{itemize}
    \item \textbf{Giáo viên/Giảng viên}: Cần công cụ tạo slide bài giảng nhanh chóng từ tài liệu có sẵn, tạo bài kiểm tra trắc nghiệm tự động
    \item \textbf{Sinh viên/Học sinh}: Cần công cụ tóm tắt tài liệu, hỏi đáp với nội dung học tập
    \item \textbf{Nhân viên văn phòng}: Cần phân tích báo cáo, tạo bản trình bày từ tài liệu dự án
    \item \textbf{Nghiên cứu sinh}: Cần công cụ phân tích, tổng hợp nhiều tài liệu nghiên cứu
\end{itemize}

\subsection{Khảo sát các hệ thống tương tự}
\label{subsection:2.1.2}

\begin{table}[H]
\centering{}
\begin{tabular}{|l|c|c|c|c|c|}
\hline
\textbf{Tiêu chí} & \textbf{ChatGPT (OpenAI)} & \textbf{Google Gemini} & \textbf{Gamma.app} & \textbf{Beautiful.ai} & \textbf{AI NVCB} \\
\hline
\textbf{Phân tích tài liệu} & $\checkmark$ (giới hạn) & $\checkmark$ & $\times$ & $\times$ & $\checkmark$ (đa định dạng) \\
\hline
\textbf{Tạo slide tự động} & $\times$ & $\times$ & $\checkmark$ & $\checkmark$ & $\checkmark$ \\
\hline
\textbf{Tạo quiz trắc nghiệm} & $\checkmark$ (thủ công) & $\checkmark$ (thủ công) & $\times$ & $\times$ & $\checkmark$ (tự động) \\
\hline
\textbf{Hỏi đáp tài liệu (RAG)} & $\checkmark$ (có phí) & $\checkmark$ & $\times$ & $\times$ & $\checkmark$ \\
\hline
\textbf{Triển khai nội bộ} & $\times$ & $\times$ & $\times$ & $\times$ & $\checkmark$ \\
\hline
\textbf{Tùy chỉnh model AI} & $\times$ & $\times$ & $\times$ & $\times$ & $\checkmark$ \\
\hline
\textbf{Chi phí} & Cao & Trung bình & Cao & Cao & Miễn phí/Thấp \\
\hline
\textbf{Bảo mật dữ liệu} & Cloud & Cloud & Cloud & Cloud & On-premise \\
\hline
\textbf{Hỗ trợ tiếng Việt} & Tốt & Tốt & Trung bình & Kém & Tốt \\
\hline
\textbf{Xuất file PPTX} & $\times$ & $\times$ & $\checkmark$ & $\checkmark$ & $\checkmark$ \\
\hline
\end{tabular}
\caption{Bảng so sánh các hệ thống tương tự}
\label{tab:2.1_compare_systems}
\end{table}

\subsection{Đánh giá ưu nhược điểm các hệ thống hiện có}
\label{subsection:2.1.3}

\textbf{ChatGPT/Google Gemini:}
\begin{itemize}
    \item \textit{Ưu điểm}: Khả năng xử lý ngôn ngữ tự nhiên mạnh mẽ, giao diện thân thiện
    \item \textit{Nhược điểm}: Chi phí cao, dữ liệu được gửi lên cloud, không tạo slide trực tiếp, giới hạn kích thước file
\end{itemize}

\textbf{Gamma.app/Beautiful.ai:}
\begin{itemize}
    \item \textit{Ưu điểm}: Tạo slide đẹp, nhiều template
    \item \textit{Nhược điểm}: Không phân tích tài liệu, không hỏi đáp, chi phí cao, không triển khai nội bộ
\end{itemize}

\subsection{Tính năng cần phát triển cho AI NVCB}
\label{subsection:2.1.4}

Dựa trên khảo sát, hệ thống AI NVCB cần đáp ứng các tính năng chính:

\begin{enumerate}
    \item \textbf{Phân tích tài liệu đa định dạng}: Hỗ trợ PDF, DOCX, TXT, MD
    \item \textbf{Hỏi đáp thông minh (RAG)}: Trả lời câu hỏi dựa trên nội dung tài liệu
    \item \textbf{Tạo slide tự động}: Sinh bản trình bày PowerPoint từ nội dung
    \item \textbf{Tạo bài trắc nghiệm}: Tự động tạo câu hỏi kiểm tra từ tài liệu
    \item \textbf{Quản lý model AI}: Cho phép thay đổi model LLM linh hoạt
    \item \textbf{Triển khai on-premise}: Bảo mật dữ liệu, không cần internet
\end{enumerate}

\section{Tổng quan chức năng}
\label{section:2.2}

\subsection{Biểu đồ use case tổng quát}
\label{subsection:2.2.1}

\begin{figure}[H]
    \centering
    \includegraphics{Hinhve/Picture1.png}
    \caption{Biểu đồ use case tổng quát}
    \label{fig:uc_overview}
\end{figure}

\textbf{Các tác nhân (Actors):}

\begin{table}[H]
\centering{}
\begin{tabular}{|l|l|p{9cm}|}
\hline
\textbf{Tác nhân} & \textbf{Vai trò} & \textbf{Mô tả} \\
\hline
\textbf{Người dùng (User)} & Tác nhân chính & Sử dụng các chức năng phân tích tài liệu, tạo slide, tạo quiz, hỏi đáp với hệ thống \\
\hline
\textbf{Quản trị viên (Admin)} & Tác nhân quản trị & Quản lý model AI, cấu hình hệ thống, theo dõi hiệu suất \\
\hline
\end{tabular}
\caption{Các tác nhân tham gia hệ thống}
\label{tab:2.2_actors}
\end{table}

\textbf{Mô tả các use case chính:}

\begin{table}[H]
\centering{}
\begin{tabular}{|l|p{10cm}|}
\hline
\textbf{Use Case} & \textbf{Mô tả} \\
\hline
\textbf{UC01: Phân tích tài liệu} & Người dùng tải lên tài liệu để hệ thống phân tích, tóm tắt và hỏi đáp \\
\hline
\textbf{UC02: Tạo Slide AI} & Hệ thống tự động tạo bản trình bày PowerPoint từ nội dung tài liệu \\
\hline
\textbf{UC03: Tạo bài trắc nghiệm} & Tự động sinh câu hỏi trắc nghiệm từ nội dung tài liệu đã tải lên \\
\hline
\textbf{UC04: Quản lý Model AI} & Quản trị viên quản lý, cài đặt và cấu hình các model LLM \\
\hline
\end{tabular}
\caption{Mô tả các use case chính}
\label{tab:2.2_main_usecases}
\end{table}

\subsection{Biểu đồ use case phân rã - Phân tích tài liệu}
\label{subsection:2.2.2}

\begin{figure}[H]
    \centering
    \includegraphics{Hinhve/Picture2.png}
    \caption{Biểu đồ use case phân rã - Phân tích tài liệu}
    \label{fig:uc_doc_analysis}
\end{figure}

\textbf{Mô tả các use case phân rã:}

\begin{table}[H]
\centering{}
\begin{tabular}{|l|p{10cm}|}
\hline
\textbf{Use Case} & \textbf{Mô tả} \\
\hline
\textbf{UC01.1: Tải tài liệu} & Người dùng tải lên 1 hoặc nhiều tài liệu (PDF, DOCX, TXT, MD) \\
\hline
\textbf{UC01.2: Tóm tắt nội dung} & Hệ thống AI tự động tóm tắt nội dung chính của tài liệu \\
\hline
\textbf{UC01.3: Hỏi đáp tài liệu (Q\&A)} & Người dùng đặt câu hỏi và nhận câu trả lời dựa trên nội dung tài liệu (RAG) \\
\hline
\textbf{UC01.4: Xem lịch sử hội thoại} & Xem lại các câu hỏi và trả lời trước đó \\
\hline
\end{tabular}
\caption{Mô tả các use case phân rã (Phân tích tài liệu)}
\label{tab:2.2_uc_doc_analysis}
\end{table}

\subsection{Biểu đồ use case phân rã - Tạo Slide AI}
\label{subsection:2.2.3}

\begin{figure}[H]
    \centering
    \includegraphics{Hinhve/Picture3.png}
    \caption{Biểu đồ use case phân rã - Tạo Slide AI}
    \label{fig:uc_slide_generation}
\end{figure}

\textbf{Mô tả các use case phân rã:}

\begin{table}[H]
\centering{}
\begin{tabular}{|l|p{10cm}|}
\hline
\textbf{Use Case} & \textbf{Mô tả} \\
\hline
\textbf{UC02.1: Nhập chủ đề/tải tài liệu} & Người dùng nhập chủ đề hoặc tải tài liệu làm nguồn nội dung \\
\hline
\textbf{UC02.2: Cấu hình số lượng slide} & Chọn số lượng slide mong muốn (1-20 slide) \\
\hline
\textbf{UC02.3: Chọn model AI} & Lựa chọn model AI phù hợp để tạo nội dung \\
\hline
\textbf{UC02.4: Tạo và xem trước slide} & Hệ thống tạo slide và hiển thị preview \\
\hline
\textbf{UC02.5: Tải xuống file PPTX} & Tải file PowerPoint về máy \\
\hline
\end{tabular}
\caption{Mô tả các use case phân rã (Tạo Slide AI)}
\label{tab:2.2_uc_slide_generation}
\end{table}

\subsection{Biểu đồ use case phân rã - Tạo bài trắc nghiệm}
\label{subsection:2.2.4}

\begin{figure}[H]
    \centering
    \includegraphics{Hinhve/Picture4.png}
    \caption{Biểu đồ use case phân rã - Tạo bài trắc nghiệm}
    \label{fig:uc_quiz_generation}
\end{figure}

\textbf{Mô tả các use case phân rã:}

\begin{table}[H]
\centering{}
\begin{tabular}{|l|p{10cm}|}
\hline
\textbf{Use Case} & \textbf{Mô tả} \\
\hline
\textbf{UC03.1: Tải tài liệu nguồn} & Tải lên 1 hoặc nhiều tài liệu làm nguồn tạo câu hỏi \\
\hline
\textbf{UC03.2: Cấu hình số câu hỏi} & Chọn số lượng câu hỏi (5-20 câu) \\
\hline
\textbf{UC03.3: Chọn độ khó} & Chọn mức độ: Dễ, Trung bình, Khó \\
\hline
\textbf{UC03.4: Tạo bài trắc nghiệm} & Hệ thống AI tự động tạo câu hỏi và đáp án \\
\hline
\textbf{UC03.5: Xem kết quả và đáp án} & Hiển thị bài trắc nghiệm với đáp án đúng \\
\hline
\end{tabular}
\caption{Mô tả các use case phân rã (Tạo bài trắc nghiệm)}
\label{tab:2.2_uc_quiz_generation}
\end{table}

\subsection{Biểu đồ use case phân rã - Quản lý Model AI}
\label{subsection:2.2.5}

\begin{figure}[H]
    \centering
    \includegraphics{Hinhve/Picture5.png}
    \caption{Biểu đồ use case phân rã - Quản lý Model AI}
    \label{fig:uc_model_management}
\end{figure}

\textbf{Mô tả các use case phân rã:}

\begin{table}[H]
\centering{}
\begin{tabular}{|l|p{10cm}|}
\hline
\textbf{Use Case} & \textbf{Mô tả} \\
\hline
\textbf{UC04.1: Xem danh sách model} & Hiển thị tất cả model đã cài đặt với thông tin chi tiết \\
\hline
\textbf{UC04.2: Tải model mới (Pull)} & Tải model mới từ Ollama registry \\
\hline
\textbf{UC04.3: Chọn model mặc định} & Đặt model mặc định cho hệ thống \\
\hline
\textbf{UC04.4: Xóa model} & Gỡ bỏ model không sử dụng để giải phóng bộ nhớ \\
\hline
\textbf{UC04.5: Cấu hình System Prompt} & Tùy chỉnh hành vi AI thông qua system prompt \\
\hline
\end{tabular}
\caption{Mô tả các use case phân rã (Quản lý Model AI)}
\label{tab:2.2_uc_model_management}
\end{table}

\subsection{Quy trình nghiệp vụ}
\label{subsection:2.2.6}

\subsubsection*{Quy trình 1: Tạo bài giảng từ tài liệu}

\begin{figure}[H]
    \centering
    \includegraphics{Hinhve/Picture6.png}
    \caption{Quy trình nghiệp vụ: Tạo bài giảng từ tài liệu}
    \label{fig:biz_process_lecture_from_doc}
\end{figure}

\textbf{Mô tả quy trình:}
\begin{enumerate}
    \item Giáo viên tải tài liệu bài giảng (PDF, DOCX) lên hệ thống
    \item Hệ thống phân tích và tóm tắt nội dung chính
    \item Nếu cần bổ sung thông tin, sử dụng chức năng hỏi đáp Q\&A
    \item Tạo slide bài giảng tự động từ nội dung đã phân tích
    \item Tải xuống file PowerPoint
    \item Tạo bài trắc nghiệm để kiểm tra học viên
\end{enumerate}

\section{Đặc tả chức năng}
\label{section:2.3}

\subsection{Đặc tả use case: Tải và phân tích tài liệu (UC01.1, UC01.2)}
\label{subsection:2.3.1}

\begin{table}[H]
\centering{}
\begin{tabular}{|l|p{11cm}|}
\hline
\textbf{Thông tin} & \textbf{Chi tiết} \\
\hline
\textbf{Tên use case} & Tải và phân tích tài liệu \\
\hline
\textbf{Mã use case} & UC01 \\
\hline
\textbf{Tác nhân} & Người dùng \\
\hline
\textbf{Mô tả} & Người dùng tải lên tài liệu và hệ thống phân tích, tóm tắt nội dung \\
\hline
\end{tabular}
\caption{Thông tin use case UC01: Tải và phân tích tài liệu}
\label{tab:2.3_uc01_info}
\end{table}

\textbf{Tiền điều kiện:}
\begin{itemize}
    \item Hệ thống đang hoạt động bình thường
    \item Model AI đã được cấu hình và sẵn sàng
    \item File tài liệu có định dạng hỗ trợ (PDF, DOCX, TXT, MD)
\end{itemize}

\textbf{Luồng sự kiện chính:}
\begin{enumerate}
    \item Người dùng truy cập trang "Phân tích tài liệu"
    \item Người dùng chọn file tài liệu từ máy tính (1 hoặc nhiều file)
    \item Hệ thống kiểm tra định dạng và kích thước file
    \item Hệ thống tải file lên server và lưu trữ tạm thời
    \item Người dùng chọn loại phân tích "Tóm tắt"
    \item Hệ thống trích xuất nội dung văn bản từ tài liệu
    \item Hệ thống gửi nội dung đến model AI để phân tích
    \item Model AI xử lý và tạo bản tóm tắt
    \item Hệ thống hiển thị kết quả tóm tắt cho người dùng
    \item Hệ thống lưu document\_id để sử dụng cho các thao tác tiếp theo
\end{enumerate}

\textbf{Luồng sự kiện phát sinh:}

\begin{table}[H]
\centering{}
\begin{tabular}{|l|p{4cm}|p{7cm}|}
\hline
\textbf{Bước} & \textbf{Điều kiện} & \textbf{Xử lý} \\
\hline
3a & File không đúng định dạng & Hiển thị thông báo lỗi, yêu cầu chọn file khác \\
\hline
3b & File vượt quá kích thước cho phép & Hiển thị cảnh báo, đề xuất chia nhỏ file \\
\hline
7a & Model AI không phản hồi & Hiển thị lỗi kết nối, đề xuất thử lại hoặc đổi model \\
\hline
8a & Nội dung chứa ký tự không hợp lệ & Hệ thống tự động làm sạch (sanitize) nội dung \\
\hline
\end{tabular}
\caption{Luồng sự kiện phát sinh của UC01}
\label{tab:2.3_uc01_exceptions}
\end{table}

\textbf{Hậu điều kiện:}
\begin{itemize}
    \item Tài liệu được lưu trữ trong hệ thống với document\_id duy nhất
    \item Bản tóm tắt được hiển thị và lưu vào lịch sử
    \item Người dùng có thể tiếp tục hỏi đáp với tài liệu
\end{itemize}

\subsection{Đặc tả use case: Hỏi đáp tài liệu (UC01.3)}
\label{subsection:2.3.2}

\begin{table}[H]
\centering{}
\begin{tabular}{|l|p{11cm}|}
\hline
\textbf{Thông tin} & \textbf{Chi tiết} \\
\hline
\textbf{Tên use case} & Hỏi đáp tài liệu (Q\&A với RAG) \\
\hline
\textbf{Mã use case} & UC01.3 \\
\hline
\textbf{Tác nhân} & Người dùng \\
\hline
\textbf{Mô tả} & Người dùng đặt câu hỏi và nhận câu trả lời dựa trên nội dung tài liệu \\
\hline
\end{tabular}
\caption{Thông tin use case UC01.3: Hỏi đáp tài liệu}
\label{tab:2.3_uc013_info}
\end{table}

\textbf{Tiền điều kiện:}
\begin{itemize}
    \item Tài liệu đã được tải lên và phân tích (UC01.1 hoàn thành)
    \item Document\_id hợp lệ tồn tại trong hệ thống
\end{itemize}

\textbf{Luồng sự kiện chính:}
\begin{enumerate}
    \item Người dùng nhập câu hỏi vào ô chat
    \item Hệ thống nhận câu hỏi và document\_id hiện tại
    \item Hệ thống sử dụng kỹ thuật RAG (Retrieval-Augmented Generation):
    \begin{itemize}
        \item Tạo embedding cho câu hỏi
        \item Tìm kiếm các đoạn văn bản liên quan từ vector store
        \item Kết hợp câu hỏi với context để tạo prompt
    \end{itemize}
    \item Gửi prompt đến model AI
    \item Model AI phân tích và tạo câu trả lời
    \item Hệ thống hiển thị câu trả lời cho người dùng
    \item Lưu câu hỏi và câu trả lời vào lịch sử hội thoại
\end{enumerate}

\textbf{Luồng sự kiện phát sinh:}

\begin{table}[H]
\centering{}
\begin{tabular}{|l|p{4cm}|p{7cm}|}
\hline
\textbf{Bước} & \textbf{Điều kiện} & \textbf{Xử lý} \\
\hline
2a & Document\_id không hợp lệ & Yêu cầu tải lại tài liệu \\
\hline
3a & Không tìm thấy context liên quan & Trả lời dựa trên kiến thức chung với disclaimer \\
\hline
5a & Câu trả lời chứa ký tự không mong muốn & Tự động sanitize output \\
\hline
\end{tabular}
\caption{Luồng sự kiện phát sinh của UC01.3}
\label{tab:2.3_uc013_exceptions}
\end{table}

\textbf{Hậu điều kiện:}
\begin{itemize}
    \item Câu trả lời được hiển thị trong giao diện chat
    \item Lịch sử hội thoại được cập nhật
    \item Người dùng có thể tiếp tục đặt câu hỏi
\end{itemize}

\subsection{Đặc tả use case: Tạo Slide AI (UC02)}
\label{subsection:2.3.3}

\begin{table}[H]
\centering{}
\begin{tabular}{|l|p{11cm}|}
\hline
\textbf{Thông tin} & \textbf{Chi tiết} \\
\hline
\textbf{Tên use case} & Tạo Slide AI \\
\hline
\textbf{Mã use case} & UC02 \\
\hline
\textbf{Tác nhân} & Người dùng \\
\hline
\textbf{Mô tả} & Tự động tạo bản trình bày PowerPoint từ chủ đề hoặc tài liệu \\
\hline
\end{tabular}
\caption{Thông tin use case UC02: Tạo Slide AI}
\label{tab:2.3_uc02_info}
\end{table}

\textbf{Tiền điều kiện:}
\begin{itemize}
    \item Model AI đã sẵn sàng
    \item Người dùng có chủ đề hoặc tài liệu nguồn
\end{itemize}

\textbf{Luồng sự kiện chính:}
\begin{enumerate}
    \item Người dùng truy cập trang "Tạo Slide AI"
    \item Người dùng nhập chủ đề slide HOẶC tải tài liệu nguồn
    \item Người dùng cấu hình số lượng slide (mặc định: 5, tối đa: 20)
    \item Người dùng chọn model AI (tùy chọn)
    \item Người dùng nhấn nút "Tạo Slide"
    \item Hệ thống xử lý:
    \begin{itemize}
        \item Nếu có tài liệu: Trích xuất nội dung
        \item Tạo prompt với cấu trúc JSON yêu cầu
        \item Gửi đến model AI
    \end{itemize}
    \item Model AI tạo nội dung slide theo cấu trúc JSON
    \item Hệ thống parse JSON và validate nội dung
    \item Hệ thống gọi PowerPointGenerator để tạo file PPTX
    \item Hiển thị preview nội dung slide
    \item Người dùng tải xuống file PPTX
\end{enumerate}

\textbf{Luồng sự kiện phát sinh:}

\begin{table}[H]
\centering{}
\begin{tabular}{|l|p{5cm}|p{6cm}|}
\hline
\textbf{Bước} & \textbf{Điều kiện} & \textbf{Xử lý} \\
\hline
7a & JSON response không hợp lệ & Retry với prompt cải tiến \\
\hline
8a & Nội dung slide có ký tự lỗi & Sanitize và tiếp tục \\
\hline
9a & Lỗi tạo file PPTX & Hiển thị lỗi, đề xuất thử lại \\
\hline
\end{tabular}
\caption{Luồng sự kiện phát sinh của UC02}
\label{tab:2.3_uc02_exceptions}
\end{table}

\textbf{Hậu điều kiện:}
\begin{itemize}
    \item File PPTX được tạo và lưu trong thư mục output/slides
    \item Người dùng có thể tải xuống file
    \item Log được ghi nhận cho việc theo dõi
\end{itemize}

	extbf{Biểu đồ hoạt động:}

\begin{figure}[H]
    \centering
    \includegraphics{Hinhve/Picture7.png}
    \caption{Biểu đồ hoạt động use case UC02: Tạo Slide AI}
    \label{fig:activity_uc02}
\end{figure}

\subsection{Đặc tả use case: Tạo bài trắc nghiệm (UC03)}
\label{subsection:2.3.4}

\begin{table}[H]
\centering{}
\begin{tabular}{|l|p{11cm}|}
\hline
\textbf{Thông tin} & \textbf{Chi tiết} \\
\hline
\textbf{Tên use case} & Tạo bài trắc nghiệm \\
\hline
\textbf{Mã use case} & UC03 \\
\hline
\textbf{Tác nhân} & Người dùng \\
\hline
\textbf{Mô tả} & Tự động tạo câu hỏi trắc nghiệm từ nội dung tài liệu \\
\hline
\end{tabular}
\caption{Thông tin use case UC03: Tạo bài trắc nghiệm}
\label{tab:2.3_uc03_info}
\end{table}

\textbf{Tiền điều kiện:}
\begin{itemize}
    \item Tài liệu nguồn đã sẵn sàng
    \item Model AI đang hoạt động
\end{itemize}

\textbf{Luồng sự kiện chính:}
\begin{enumerate}
    \item Người dùng truy cập trang "Tạo bài trắc nghiệm"
    \item Người dùng tải lên tài liệu nguồn (1 hoặc nhiều file)
    \item Người dùng cấu hình:
    \begin{itemize}
        \item Số lượng câu hỏi (5-20 câu)
        \item Độ khó (Dễ/Trung bình/Khó)
    \end{itemize}
    \item Người dùng nhấn "Tạo bài trắc nghiệm"
    \item Hệ thống trích xuất nội dung từ tài liệu
    \item Hệ thống tạo prompt yêu cầu tạo quiz với format chuẩn
    \item Model AI tạo câu hỏi, đáp án và giải thích
    \item Hệ thống parse và format kết quả
    \item Hiển thị bài trắc nghiệm với đáp án
\end{enumerate}

\textbf{Luồng sự kiện phát sinh:}

\begin{table}[H]
\centering{}
\begin{tabular}{|l|p{5cm}|p{6cm}|}
\hline
\textbf{Bước} & \textbf{Điều kiện} & \textbf{Xử lý} \\
\hline
5a & Tài liệu quá ngắn & Cảnh báo và đề xuất thêm nội dung \\
\hline
7a & Format câu hỏi không chuẩn & Sanitize và reformat tự động \\
\hline
7b & Câu hỏi không bằng tiếng Việt & Thêm system prompt bắt buộc tiếng Việt \\
\hline
\end{tabular}
\caption{Luồng sự kiện phát sinh của UC03}
\label{tab:2.3_uc03_exceptions}
\end{table}

\textbf{Hậu điều kiện:}
\begin{itemize}
    \item Bài trắc nghiệm được hiển thị với format rõ ràng
    \item Đáp án đúng được đánh dấu
    \item Người dùng có thể copy hoặc xuất kết quả
\end{itemize}

\subsection{Đặc tả use case: Quản lý Model AI (UC04)}
\label{subsection:2.3.5}

\begin{table}[H]
\centering{}
\begin{tabular}{|l|p{11cm}|}
\hline
\textbf{Thông tin} & \textbf{Chi tiết} \\
\hline
\textbf{Tên use case} & Quản lý Model AI \\
\hline
\textbf{Mã use case} & UC04 \\
\hline
\textbf{Tác nhân} & Quản trị viên \\
\hline
\textbf{Mô tả} & Quản lý các model LLM: xem, tải, xóa, cấu hình \\
\hline
\end{tabular}
\caption{Thông tin use case UC04: Quản lý Model AI}
\label{tab:2.3_uc04_info}
\end{table}

\textbf{Tiền điều kiện:}
\begin{itemize}
    \item Ollama server đang chạy
    \item Người dùng có quyền quản trị
\end{itemize}

\textbf{Luồng sự kiện chính:}
\begin{enumerate}
    \item Quản trị viên truy cập trang "Quản lý Model AI"
    \item Hệ thống hiển thị danh sách model đã cài đặt với thông tin:
    \begin{itemize}
        \item Tên model
        \item Kích thước
        \item Thời gian sửa đổi
    \end{itemize}
    \item Quản trị viên có thể:
    \begin{itemize}
        \item \textbf{Xem chi tiết}: Click vào model để xem thông tin
        \item \textbf{Đặt mặc định}: Chọn model làm mặc định cho hệ thống
        \item \textbf{Tải model mới}: Nhập tên model từ Ollama registry và pull
        \item \textbf{Xóa model}: Xóa model không sử dụng
        \item \textbf{Cấu hình System Prompt}: Thiết lập hành vi AI mặc định
    \end{itemize}
\end{enumerate}

\textbf{Luồng sự kiện phát sinh:}

\begin{table}[H]
\centering{}
\begin{tabular}{|l|p{5cm}|p{6cm}|}
\hline
\textbf{Bước} & \textbf{Điều kiện} & \textbf{Xử lý} \\
\hline
3a & Ollama server không phản hồi & Hiển thị lỗi kết nối \\
\hline
3b & Model pull thất bại & Hiển thị chi tiết lỗi, đề xuất kiểm tra tên model \\
\hline
3c & Không đủ dung lượng để tải model & Cảnh báo và đề xuất xóa model cũ \\
\hline
\end{tabular}
\caption{Luồng sự kiện phát sinh của UC04}
\label{tab:2.3_uc04_exceptions}
\end{table}

\textbf{Hậu điều kiện:}
\begin{itemize}
    \item Danh sách model được cập nhật
    \item Model mặc định được lưu vào cấu hình
    \item System prompt được áp dụng cho các request tiếp theo
\end{itemize}

\subsection{Đặc tả use case: Xem lịch sử hội thoại (UC01.4)}
\label{subsection:2.3.6}

\begin{table}[H]
\centering{}
\begin{tabular}{|l|p{11cm}|}
\hline
\textbf{Thông tin} & \textbf{Chi tiết} \\
\hline
\textbf{Tên use case} & Xem lịch sử hội thoại \\
\hline
\textbf{Mã use case} & UC01.4 \\
\hline
\textbf{Tác nhân} & Người dùng \\
\hline
\textbf{Mô tả} & Xem lại các câu hỏi và trả lời trước đó \\
\hline
\end{tabular}
\caption{Thông tin use case UC01.4: Xem lịch sử hội thoại}
\label{tab:2.3_uc014_info}
\end{table}

\textbf{Tiền điều kiện:}
\begin{itemize}
    \item Có ít nhất một cuộc hội thoại đã thực hiện
    \item Document\_id hợp lệ
\end{itemize}

\textbf{Luồng sự kiện chính:}
\begin{enumerate}
    \item Người dùng vào trang phân tích tài liệu
    \item Hệ thống tự động load lịch sử hội thoại theo document\_id
    \item Hiển thị danh sách các câu hỏi và câu trả lời theo thời gian
    \item Người dùng có thể:
    \begin{itemize}
        \item Cuộn xem các tin nhắn cũ
        \item Tiếp tục hội thoại từ context trước
    \end{itemize}
\end{enumerate}

\textbf{Hậu điều kiện:}
\begin{itemize}
    \item Lịch sử hội thoại được hiển thị đầy đủ
    \item Context được duy trì cho các câu hỏi tiếp theo
\end{itemize}

\section{Yêu cầu phi chức năng}
\label{section:2.4}

\subsection{Yêu cầu về hiệu năng}
\label{subsection:2.4.1}

\begin{table}[H]
\centering{}
\begin{tabular}{|l|p{8cm}|p{3cm}|}
\hline
\textbf{Yêu cầu} & \textbf{Mô tả} & \textbf{Chỉ tiêu} \\
\hline
\textbf{NFR01} & Thời gian phản hồi tóm tắt tài liệu & < 30 giây cho file < 10MB \\
\hline
\textbf{NFR02} & Thời gian tạo slide & < 60 giây cho 10 slide \\
\hline
\textbf{NFR03} & Thời gian tạo quiz & < 45 giây cho 10 câu hỏi \\
\hline
\textbf{NFR04} & Số người dùng đồng thời & Tối thiểu 10 users \\
\hline
\textbf{NFR05} & Kích thước file tối đa & 50MB cho mỗi file \\
\hline
\end{tabular}
\caption{Yêu cầu phi chức năng về hiệu năng}
\label{tab:2.4_performance}
\end{table}

\subsection{Yêu cầu về độ tin cậy}
\label{subsection:2.4.2}

\begin{table}[H]
\centering{}
\begin{tabular}{|l|p{11cm}|}
\hline
\textbf{Yêu cầu} & \textbf{Mô tả} \\
\hline
\textbf{NFR06} & Hệ thống hoạt động 99\% uptime trong giờ làm việc \\
\hline
\textbf{NFR07} & Tự động retry khi model AI không phản hồi (tối đa 3 lần) \\
\hline
\textbf{NFR08} & Backup dữ liệu định kỳ (daily) \\
\hline
\textbf{NFR09} & Auto cleanup file tạm sau 24 giờ \\
\hline
\end{tabular}
\caption{Yêu cầu phi chức năng về độ tin cậy}
\label{tab:2.4_reliability}
\end{table}

\subsection{Yêu cầu về tính dễ dùng}
\label{subsection:2.4.3}

\begin{table}[H]
\centering{}
\begin{tabular}{|l|p{11cm}|}
\hline
\textbf{Yêu cầu} & \textbf{Mô tả} \\
\hline
\textbf{NFR10} & Giao diện thân thiện, responsive trên desktop \\
\hline
\textbf{NFR11} & Hỗ trợ hoàn toàn tiếng Việt (UI và output) \\
\hline
\textbf{NFR12} & Hiển thị trạng thái xử lý realtime (loading, progress) \\
\hline
\textbf{NFR13} & Thông báo lỗi rõ ràng, có hướng dẫn khắc phục \\
\hline
\end{tabular}
\caption{Yêu cầu phi chức năng về tính dễ dùng}
\label{tab:2.4_usability}
\end{table}

\subsection{Yêu cầu về bảo mật}
\label{subsection:2.4.4}

\begin{table}[H]
\centering{}
\begin{tabular}{|l|p{11cm}|}
\hline
\textbf{Yêu cầu} & \textbf{Mô tả} \\
\hline
\textbf{NFR14} & Triển khai on-premise, dữ liệu không gửi ra ngoài \\
\hline
\textbf{NFR15} & File tải lên được lưu trữ tạm và tự động xóa \\
\hline
\textbf{NFR16} & API có CORS protection \\
\hline
\textbf{NFR17} & Không log nội dung nhạy cảm của tài liệu \\
\hline
\end{tabular}
\caption{Yêu cầu phi chức năng về bảo mật}
\label{tab:2.4_security}
\end{table}

\subsection{Yêu cầu về tính dễ bảo trì}
\label{subsection:2.4.5}

\begin{table}[H]
\centering{}
\begin{tabular}{|l|p{11cm}|}
\hline
\textbf{Yêu cầu} & \textbf{Mô tả} \\
\hline
\textbf{NFR18} & Code được tổ chức theo kiến trúc phân lớp (Frontend/Backend/Services) \\
\hline
\textbf{NFR19} & Logging đầy đủ cho debugging và monitoring \\
\hline
\textbf{NFR20} & Hỗ trợ hot-reload cho development \\
\hline
\textbf{NFR21} & Docker containerization cho deployment dễ dàng \\
\hline
\end{tabular}
\caption{Yêu cầu phi chức năng về tính dễ bảo trì}
\label{tab:2.4_maintainability}
\end{table}

\subsection{Yêu cầu về công nghệ}
\label{subsection:2.4.6}

\begin{table}[H]
\centering{}
\begin{tabular}{|l|l|l|}
\hline
\textbf{Thành phần} & \textbf{Công nghệ} & \textbf{Phiên bản} \\
\hline
\textbf{Backend Framework} & FastAPI & 0.115.0+ \\
\hline
\textbf{Frontend Framework} & Streamlit & 1.40.0+ \\
\hline
\textbf{AI/LLM Framework} & LangChain + Ollama & Latest \\
\hline
\textbf{Vector Database} & FAISS & Latest \\
\hline
\textbf{Embedding Model} & sentence-transformers & all-MiniLM-L6-v2 \\
\hline
\textbf{Database} & SQLite/PostgreSQL & - \\
\hline
\textbf{Cache} & Redis (optional) & - \\
\hline
\textbf{Container} & Docker & 20.0+ \\
\hline
\textbf{Reverse Proxy} & Nginx & Latest \\
\hline
\textbf{Programming Language} & Python & 3.11+ \\
\hline
\end{tabular}
\caption{Yêu cầu về công nghệ}
\label{tab:2.4_technology}
\end{table}

\section*{Kết luận chương}

Chương 2 đã trình bày chi tiết quá trình khảo sát hiện trạng và phân tích yêu cầu cho hệ thống AI NVCB. Qua việc khảo sát các hệ thống tương tự trên thị trường (ChatGPT, Google Gemini, Gamma.app, Beautiful.ai), đã xác định được các tính năng cần phát triển để tạo nên lợi thế cạnh tranh: phân tích tài liệu đa định dạng, tạo slide tự động, tạo quiz trắc nghiệm, và khả năng triển khai on-premise.

Biểu đồ use case tổng quát và các biểu đồ phân rã đã mô tả đầy đủ các chức năng của hệ thống với hai tác nhân chính: Người dùng và Quản trị viên. Đặc tả chi tiết 6 use case quan trọng nhất đã làm rõ luồng sự kiện chính, luồng phát sinh, tiền điều kiện và hậu điều kiện cho từng chức năng.

Các yêu cầu phi chức năng về hiệu năng, độ tin cậy, tính dễ dùng, bảo mật, bảo trì và công nghệ đã được xác định cụ thể với các chỉ tiêu đo lường (NFR01-NFR21). Bảng yêu cầu công nghệ tại mục 2.4.6 đã liệt kê sơ bộ các công nghệ dự kiến sử dụng, sẽ được phân tích chi tiết và giải thích lý do lựa chọn trong Chương 3.

Các yêu cầu chức năng (UC01-UC04) và phi chức năng (NFR01-NFR21) được xác định trong chương này là cơ sở để tiến hành lựa chọn công nghệ (Chương 3), thiết kế kiến trúc và triển khai hệ thống (Chương 4).

\end{document}
